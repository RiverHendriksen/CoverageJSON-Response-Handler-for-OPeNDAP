\documentclass[letterpaper,10pt,draftclsnofoot,onecolumn]{IEEEtran}
\usepackage{url}
\usepackage{setspace}
\usepackage{multirow}

\usepackage{geometry}
\geometry{textheight=9.5in, textwidth=7in}

\def \Author{Corey Hemphill}
\def \Assignment{Problem Statement}
\def \HomeworkDueDate{October 9, 2017}

\def \DocType{	%Problem Statement
		%Requirements Document
		%Technology Review
		%Design Document
		Senior Capstone
		}

\newcommand{\NameSigPair}[1]{\par
\makebox[2.75in][r]{#1} \hfil \makebox[3.25in]{\makebox[2.25in]{\hrulefill} \hfill \makebox[.75in]{\hrulefill}}
\par\vspace{-12pt} \textit{\tiny\noindent
\makebox[2.75in]{} \hfil \makebox[3.25in]{\makebox[2.25in][r]{Signature} \hfill	\makebox[.75in][r]{Date}}}}


\begin{document}
\begin{titlepage}
    \pagenumbering{gobble}
    \begin{singlespace}
        \hfill  
        \par\vspace{.2in}
        \centering
        \scshape{
            \huge  \DocType \par
           	\huge cs461 Fall 2017 \par
            {\large\today}\par
            \vspace{.5in}
            \textbf{\Huge\Assignment}\par
            \vspace{.5in}
            {\large Prepared by }\par
           	\textbf{\Author}\par
            \vspace{5pt}
            }
            \vspace{120pt}
        \begin{abstract}
        OPeNDAP is a free and open source advanced software for remote data retrieval that
		provides a way for scientists to share data across the internet, and it is widely
		used in the earth-science research community. OPeNDAP currently supports a large
		number of different data formats such as HDF4, HDF5, NetCDF 3, FreeForm,
		and NcML, however, it does not currently contain a handler for the CoverageJSON
		data format. In an effort to further the usefulness and robustness of OPeNDAP,
		senior capstone team members working on this project will work to implement a
		CoverageJSON handler for OPeNDAP.
        \end{abstract} 
    \end{singlespace}
\end{titlepage}
\newpage
\section{Problem Statement}
We live at a time where getting information is really easy, and it is also relatively easy to collect and store large amounts of data if you have the capital to do so. The problem with storing so much data is that it sometimes becomes more difficult to access due to formatting inconsistencies and a number of other factors. NASA collects an extremely large amount of data constantly from its many active missions, and all of that data ends up stored in databases, waiting to be put to use by scientists, researchers, software developers, engineers, etc. In order to access this data and put it to use, software tools like OPeNDAP are needed to provide a discipline-neutral means of sharing that data on the internet.
\section{Proposed Solution}
OPeNDAP’s goal is to allow users to immediately access the data they need in a format they need. Although OPeNDAP supports a number of representation formats such as for HDF4, HDF5, NetCDF 3, FreeForm, and NcML, it does not yet support CoverageJSON. CoverageJSON (CovJSON) is an emerging data model and representation format for coverage data. The CovJSON data model is intended to be used by software developers to extract coverage data that is collected by NASA satellites and then stored on servers. Allowing users the ability to easily and immediately access useful scientific data enables the creation of powerful web applications such as NASA JPL SOTO. https://podaac-tools.jpl.nasa.gov/soto 
The primary objective of the project is to work with NASA JPL and the OPeNDAP team to develop a response handler for CovJSON within OPeNDAP using C++. The secondary objective of the project is to further evangelize CovJSON and OPeNDAP as a preferred method of sharing coverage data over the web. If we want developers to use tools like OPeNDAP and CovJSON, we need to let them know they exist. In addition to developing the CovJSON response handler, the team will also develop a project website on GitHub with tests and documentation and a poster to promote the project.
By implementing a response handler for CovJSON in OPeNDAP, we expand access to all of the amazing data that NASA has collected over the years. Allowing people immediate access to scientific data leads to amazing things. The problem we want to eliminate is the inherent difficulty that exists in obtaining useful, relevant data. As developers, we want to spend less time and energy finding the resources we need, and more time basking in our program’s glory.

\end{document}
