\documentclass[letterpaper,10pt,draftclsnofoot,onecolumn]{IEEEtran}
\usepackage{graphicx, geometry, hyperref, geometry, listings, enumitem, balance, longtable, url, color, float, alltt, amsthm, amsmath, amssymb}
\geometry{margin=.75in}
\hypersetup{ %PDF Metadata setup
  colorlinks = true,
  urlcolor = black,
  pdfauthor = {Riley Rimer},
  pdfkeywords = {CS 462 Problem Statement},
  pdftitle = {CS 462 Problem Statement},
  pdfsubject = {CS 462 Problem Statement},
  pdfpagemode = UseNone
}

\title{
Fall 2017 CS 462 Problem Statement\\
\large CoverageJSON Response Handler for OPeNDAP
}
\author{Riley Rimer}
\date{October 5, 2017}

\begin{document}
\begin{titlepage}

\maketitle
\centering

\begin{abstract}
The purpose of our project is add on the already existing data response handlers of the OPeNDAP network data access protocol. OPeNDAP currently does not have provide data access handlers for a coverage data format, which is a structure that associates positions in space and time with corresponding data values. OPeNDAP is currently used by NASA for their satellite data, and the addition of a coverage data format would be greatly beneficial due to its similarity with satellite data. The main deliverable of this project will be a data response handler for the CoverageJSON data format, which fulfills this need. The solution will need to include thorough documentation and testing so that it can eventually be incorporated into OPeNDAP project. Prior to the completion of this project our group will need to gain extensive knowledge on OPeNDAP and how it handles data responses, along with knowledge of the CoverageJSON data format.
\end{abstract}

\end{titlepage}

\section{Problem Statement}
One way that satellite data is currently handled by NASA is through OPeNDAP, which stands for "Open-source Project for a Network Data Access Protocol". OPeNDAP is a protocol that allows users to pull data from it in multiple different formats including ASCII, netCDF3, netCDF4 and binary (DAP2) object serializations. OPeNDAP does not have any data response handlers to supply data in CoverageJSON, which is a JSON based format used for encoding coverage data just like what is collected by satellites. CoverageJSON would be particularly useful as a data format available from OPeNDAP considering the added capability of pulling usable coverage data from the satellites rather than data in the more scientific based models like what is currently implemented in OPeNDAP. Coverage data can be useful since it is a format that associates positions in space and time. The purpose of this project is to add a data response handler to OPenDAP for CoverageJSON. Adding in a data response handler for CoverageJSON would greatly help improve data interpretation for data with coordinates in space and time and help the NASA and other consumers of OPenDAP incorporate coverage in their work flows and publications.

\section{Proposed Solution}
In short the solution would be to create an OPeNDAP CovJSON response handler. This would involve our team first getting an extensive understanding of OPeNDAP and how response handling works within it. We would also need to learn the CoverageJSON data format in depth. Once our group has a good understanding of how this protocol and data format can fit together we need to plan out the actual implementation of the CoverageJSON response handler for OPeNDAP. This implementation will need to be have accompanying documentation and testing that is thorough enough for it to be brought into the OPeNDAP project. The project implementation will written in C++ with the purpose of being brought into the OPeNDAP codebase in mind the entire time. The exact proposed deliverables include: A website for the project served from Github, the OPeNDAP CoverageJSON response handler, documentation and tests and a poster for the project to the AGU(Agricultural Geophysical Union).

\section{Performance Metrics}
The success of this project will be measured primarily on the state of the response handler made by the end of the project period. Several factors that will need to be weighed on the implementation are: completion of the C++ response handler, documentation, tests, if it is feasible for it to be brought into the OPeNDAP code base, the project website and the poster to the AGU(Agricultural Geophysical Union). Ideally for a complete success of this project the CoverageJSON response handler will be fully completed at the end of the project, along with full documentation up the standards reasonable for the OPeNDAP project and thorough enough testing for it to be included in OPeNDAP. The best case would be the project already in the process of being incorporated into the OPeNDAP project.

\end{document}