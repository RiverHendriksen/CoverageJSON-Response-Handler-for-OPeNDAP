\documentclass[onecolumn, draftclsnofoot,10pt, compsoc]{IEEEtran}
\usepackage{graphicx}
\usepackage{url}
\usepackage{setspace}

\usepackage{geometry}
\geometry{textheight=9.5in, textwidth=7in}

\def \CapstoneTeamNumber{55}
\def \GroupMemberOne{Riley Rimer}
\def \GroupMemberTwo{River Hendriksen}
\def \GroupMemberThree{Corey Hemphill}
\def \CapstoneProjectName{CoverageJSON Response Handler for OPeNDAP}
\def \CapstoneSponsorCompany{NASA Jet Propulsion Laboratory}
\def \CapstoneSponsorPerson{Lewis John McGibbney}

\def \DocType{	%Problem Statement
				Requirements Document
				%Technology Review
				%Design Document
				%Progress Report
				}
			
\newcommand{\NameSigPair}[1]{\par
\makebox[2.75in][r]{#1} \hfil 	\makebox[3.25in]{\makebox[2.25in]{\hrulefill} \hfill		\makebox[.75in]{\hrulefill}}
\par\vspace{-12pt} \textit{\tiny\noindent
\makebox[2.75in]{} \hfil		\makebox[3.25in]{\makebox[2.25in][r]{Signature} \hfill	\makebox[.75in][r]{Date}}}}
% 3. If the document is not to be signed, uncomment the RENEWcommand below
\renewcommand{\NameSigPair}[1]{#1}

%%%%%%%%%%%%%%%%%%%%%%%%%%%%%%%%%%%%%%%
\begin{document}
\begin{titlepage}
    \pagenumbering{gobble}
    \begin{singlespace}
        \hfill    
        \par\vspace{.2in}
        \centering
        \scshape{
            \huge CS Capstone \DocType \par
            {\large\today}\par
            \vspace{.5in}
            \textbf{\Huge\CapstoneProjectName}\par
                        \vspace{.5in}

            \vfill
            {\large Prepared for}\par
            \Huge \CapstoneSponsorCompany\par
            \vspace{5pt}
            {\Large\NameSigPair{\CapstoneSponsorPerson}\par}
            {\large Prepared by }\par
            Group\CapstoneTeamNumber\par
            \vspace{5pt}
            {\Large
                \NameSigPair{\GroupMemberOne}\par
                \NameSigPair{\GroupMemberTwo}\par
                \NameSigPair{\GroupMemberThree}\par
            }
            \vspace{20pt}
        }
        \begin{abstract}
        	This document provides the requirements needed to complete the CoverageJSON Reponse Handler for OPenDAP. 
        \end{abstract}     
    \end{singlespace}
\end{titlepage}
\newpage
\pagenumbering{arabic}
%\tableofcontents
% 7. uncomment this (if applicable). Consider adding a page break.
%\listoffigures
%\listoftables
\clearpage

\section{Introduction}
\subsection{Purpose}
The purpose of this project is to further the usability of OPeNDAP by integrating it with CoverageJSON.
This implementation will be helpful for all users of both OPeNDAP and CoverageJSON but will be particularly useful to the NASA JPL.

\subsection{Scope}
The result of this project will be a CoverageJSON response handler for OPeNDAP.
This will allow for OPeNDAP to feed users data in the CoverageJSON data format, which will allow users to view their data as a coverage, rather than the scientific formats currently implemented in OPeNDAP.

\subsection{Definitions}
\begin{enumerate}
\item OPeNDAP - Open Source Project for a Network Data Access Protocol.
\item CoverageJSON - JSON data format for encoding coverage data.
\item JSON - JavaScript Object Notation.
\item NASA JPL - The National Aeronautics and Space Administration Jet Proplusion Laboratory
\end{enumerate}

\subsection{References}
\begin{enumerate}
\item OPeNDAP Advanced Software for Remote Data Retrieval https://www.opendap.org/
\end{enumerate}

%\subsection{Overview}

\section{Overall Description}
\subsection{Product Perspective}
The CoverageJSON response handler will be incorporated into the larger OPeNDAP project. Therefore all calls will be similiar to those implemented in OPeNDAP. The current OPeNDAP Javascript call looks like this: 
\\\\
\texttt{createDataRequestForm({"url" : "http://test.com/data.gz", "containerID" : "requestform"});}
\\\\The CoverageJSON handler will be similar in execution to remain faithful to the requirements set by OPeNDAP. 

\subsection{Product Functionality}
The CoverageJSON response handler will have the same functionality as the response handlers already implemented in OPeNDAP.
\begin{enumerate}
\item Data that is converted to CovJSON will contain the same information as the source, but in the CovJSON format. 
\item Users will be able to get data via a GUI that is already implemented in OPeNDAP, however there will be a new option for CovJSON. 
\item The handler will work and be implemented in OPeNDAP's Source Github. 
\end{enumerate}

\subsection{User Characteristics}
The expected user characteristics will be the same as the characteristic expected of an OPeNDAP user.
These characteristics include:
\begin{enumerate}
\item Scientists looking to share data over the Internet.
\item Groups looking to provide compatible clients, servers and SDKs.
\item Users looking to conform the to NASA community standard.

\end{enumerate}
%\subsection{Constraints}

\subsection{Assumptions and Dependencies}
\begin{enumerate}
\item There will be enough accurate documentation on OPeNDAP and its response handlers.
\item There will be a server environment capable enough to test on.
\item There will be feedback on the testing and documentation needed to have the response handler pulled into the OPeNDAP project.
\end{enumerate}

\section{Requirements}
The CoverageJSON response handler will need to fulfill the following requirements:
\begin{enumerate}
\item Handle and feed out data that is in the CoverageJSON format and a correct implementation of the starting data.
\item Able to be pulled into the OPeNDAP source project.
\item The handler will need to be able to be called in the same fashion the current OPeNDAP handlers are called, and act similarly.
\item The handler must be testable and meet the standards defined by the client. 
\end{enumerate}

\end{document}